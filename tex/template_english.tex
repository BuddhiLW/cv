%%%%%%%%%%%%%%%%%%%%%%%%%%%%%%%%%%%%%%%%%
% Twenty Seconds Resume/CV
% LaTeX Template
% Version 1.1 (8/1/17)
%
% This template has been downloaded from:
% http://www.LaTeXTemplates.com
%
% Original author:
% Carmine Spagnuolo (cspagnuolo@unisa.it) with major modifications by
% Vel (vel@LaTeXTemplates.com)
%
% License:
% The MIT License (see included LICENSE file)
%
%%%%%%%%%%%%%%%%%%%%%%%%%%%%%%%%%%%%%%%%%

% ----------------------------------------------------------------------------------------
%	PACKAGES AND OTHER DOCUMENT CONFIGURATIONS
% ----------------------------------------------------------------------------------------

\documentclass[letterpaper]{../cls/twentysecondcvenglish} % a4paper for A4
\usepackage{hyperref}

% Definitions of Pinyin
\makeatletter
\def\py@yunpriv#1{%
  \if a#1 10\else
  \if o#1 9\else
  \if e#1 8\else
  \if i#1 7\else
  \if u#1 6\else
  \if v#1 5\else
  \if A#1 4\else
  \if O#1 3\else
  \if E#1 2\fi\fi\fi\fi\fi\fi\fi\fi\fi0
}

\def\py@init{%
  \edef\py@befirst{}%
  \edef\py@char{}\edef\py@tuneletter{}%
  \def\py@last{}%
  \def\py@tune{5}%
}

% Usage:
% \pinyin{Hao3hao3\ xue2xi2} (好好学习)
\def\pinyin#1{%
  \edef\py@postscan{#1}%
  \py@init
  % scan
  \loop
  \edef\py@char{\expandafter\@car\py@postscan\@nil}%
  \edef\py@postscan{\expandafter\@cdr\py@postscan\@nil}%
  \ifnum 0 < 0\py@char
    \edef\py@tune{\py@char}%
    \py@first \py@tuneat\py@tuneletter\py@tune \py@last\kern -4sp\kern 4sp{}\py@init
  \else
    \ifnum\py@yunpriv\py@char > \py@yunpriv\py@tuneletter
      \edef\py@tuneletter{\py@char}\edef\py@first{\py@befirst}\def\py@last{}%
    \else
      \edef\py@last{\py@last\if v\py@char\"u\else\py@char\fi}%
    \fi
    \edef\py@befirst{\py@befirst\if v\py@char\"u\else\py@char\fi}%
  \fi
  \ifx\py@postscan\@empty\else
  \repeat
}

\let\py@macron \=
\let\py@acute \'
\let\py@hacek \v
\let\py@grave \`

%% \py@tuneat{Letter}{tune}
\def\py@tuneat#1#2{%
  \if v#1%
    \py@tune@v #2%
  \else
  \if i#1%
    \py@tune@i #2%
  \else
    \ifcase#2%
      \or\py@macron #1\or\py@acute #1\or\py@hacek #1\or\py@grave #1\else #1%
    \fi
  \fi\fi
}

\def\py@tune@v#1{{%
    \dimen@ii 1ex%
    \fontdimen5\font 1.1ex%
    \rlap{\"u}%
    \fontdimen5\font .6ex%
    \ifcase#1%
      \or\py@macron u\or\py@acute u\or\py@hacek u\or\py@grave u\else u%
    \fi
    \fontdimen5\font\dimen@ii
  }}

\def\py@tune@i#1{%
  \ifcase#1
    \or\py@macron \i\or\py@acute \i\or\py@hacek \i\or\py@grave \i\else i%
  \fi
}
\makeatletter
% End of pinyin

% ----------------------------------------------------------------------------------------
% PERSONAL INFORMATION
% ----------------------------------------------------------------------------------------

% If you don't need one or more of the below, just remove the content leaving the command, e.g. \cvnumberphone{}

\profilepic{W1-2} % Profile picture
\cvname{Pedro Gomes \\Branquinho} % Your name
\cvjobtitle{Engineering Physicist} % Job title/career

\cvdate{October, 07, 1997} % Date of birth
\cvaddress{Franca, São Paulo - Brazil} % Short address/location, use \newline if more than 1 line is required
\cvnumberphone{+55 16 99340-1215} % Phone number
\linkedin{\href{https://www.linkedin.com/in/pedro-g-branquinho/}{LinkedIn Profile}} % Personal website
\cvsite{\href{www.buddhilw.com}{Personal Website}} % Personal website
% \cvsite2{\href{www.buddhilw.com}{\color{violet!60!black}www.buddhilw.com}} % Personal website
\cvmail{pedro.branquinho@usp.br} % Email address

% ----------------------------------------------------------------------------------------

\begin{document}

% ----------------------------------------------------------------------------------------
% ABOUT ME
% ----------------------------------------------------------------------------------------

\aboutme{I'm a Software Developer with strong background in STEM. I have know- ledge in both Front and Backend, as well as an advanced knowledge of Statistics and Scientific Computing.

  I have worked in Industry automating billing and reports for big and small companies. Also, I have profitable solo projects with Crypto Trading Bots.}

% To have no About Me section, just remove all the text and leave \aboutme{}

% ----------------------------------------------------------------------------------------
% SKILLS
% ----------------------------------------------------------------------------------------

% Skill bar section, each skill must have a value between 0 an 6 (float)
\skills{{Emacs/5.7}, {LaTeX/5.7}, {Python/5.5}, {Go/5},{Julia/4.5},{Git/4.5},{Clojure(Script)/4.5}, {Scripting (Bash, et al.)/4}, {JavaScript/3.5}}

% ------------------------------------------------

% Skill text section, each skill must have a value between 0 an 6
% \skillstext{{lovely/4},{narcissistic/3}}

% ----------------------------------------------------------------------------------------

\makeprofile % Print the sidebar

% ----------------------------------------------------------------------------------------
% INTERESTS
% ----------------------------------------------------------------------------------------

\section{Professional Interests}

% \vspace{10mm}

\begin{itemize}
  \item Data Science - Recommender Systems, Hypothesis testing, PCA;
  \item Webapp Applications using Machine Learning;
  \item Linux System's automation and pipelines;
  \item Backend Development;
  % \item Mathematical modeling of dynamic systems (ODEs/PDEs);
  % \item Operational Reaseach algorithms (OR-Tools);
\end{itemize}
\vspace{0.2cm}
% ----------------------------------------------------------------------------------------
% EDUCATION
% ----------------------------------------------------------------------------------------

\section{Formal Education}

\begin{twenty} % Environment for a list with descriptions
  \twentyitem{2016 - 2022}{Graduation (Completed)}{\normalsize{São Paulo University, USP}}{Engineering Physics}
  % \twentyitem{<dates>}{<title>}{<location>}{<description>}
\end{twenty}
\vspace{0.2cm}
% ----------------------------------------------------------------------------------------
% PUBLICATIONS
% ----------------------------------------------------------------------------------------

\section{Eletronic Publications}

\begin{twentyshort} % Environment for a short list with no
  % descriptions
  \twentyitemshort{2021}{\href{https://github.com/BuddhiLW/commons-csv-clj}{\color{blue!50!violet}
    Industry (Flow Finance) - Automate billing and Ledger history.}}
  \twentyitemshort{2021}{\href{https://github.com/BuddhiLW/CloshBashika}{\color{blue!50!violet}
    Industry (Lupo S.A.) - Automate technical reports with Clojure/LaTeX.}}
  \twentyitemshort{2020}{\href{https://github.com/BuddhiLW/EEL_EstatMultiVarRN}{\color{blue!50!violet}Machine
    Learning, with R - while enrolled in Multivariate Statics.}}
  % \twentyitemshort{2020}{\href{https://github.com/BuddhiLW/MC-LaTeX/tree/master/LabEELw/LabEELw/MaterialMC/Livro_PT1}{\color{blue!50!violet}Introduction
  %     to \LaTeX, P1.}}
  % \twentyitemshort{2020}{\href{https://github.com/BuddhiLW/MC-LaTeX/tree/master/LabEELw/LabEELw/MaterialMC/Livro_PT2}{\color{blue!50!violet}
  %     Essential Commands in \LaTeX, P2.
  %   }}
  % \twentyitemshort{2020}{\href{https://github.com/BuddhiLW/MC-LaTeX/tree/master/LabEELw/LabEELw/MaterialMC/Livro_PT3}{\color{blue!50!violet}Bibliographical
  %     References, Citations and Beamer Presentations, P3.}}
  \twentyitemshort{2020}{\href{https://www.youtube.com/watch?v=spYCKElN-v0&list=PLweBgessvalf9eGUknfHQZ6_q2Nz7t8df}{\color{blue!50!violet}
    A Minicourse on \LaTeX{}} - Material available on YouTube (Portuguese).}
  % \twentyitemshort{<dates>}{<title/description>}
\end{twentyshort}
\vspace{0.2cm}
% ----------------------------------------------------------------------------------------
% AWARDS
% ----------------------------------------------------------------------------------------

% \section{Awards}

% \begin{twentyshort} % Environment for a short list with no descriptions
%   \twentyitemshort{1987}{All-Time Best Fantasy Novel.}
%   \twentyitemshort{1998}{All-Time Best Fantasy Novel before 1990.}
%   % \twentyitemshort{<dates>}{<title/description>}
% \end{twentyshort}

% ----------------------------------------------------------------------------------------
% EXPERIENCE
% ----------------------------------------------------------------------------------------

\section{Extracurricular Experiences}

\begin{twenty} % Environment for a list with descriptions
  \twentyitem{2022}{\textit{Programming in Python}}{Meta}{Course on Algorithms and Datastructures using Python.}
  \twentyitem{2022}{\textit{Version Control}}{Meta}{Course on Git, Github and Version Control practices.}
  \twentyitem{2022}{\textit{Introduction to Back-End Development}}{Meta}{Course on how the Web Works, HTTP, REST, APIs, TDD and best practices.}
  \twentyitem{2022}{\textit{Google Go Specialization}}{University of California (UCI)}{Three courses exploring Golang syntax, Object Orientation and Concurrency}
  \twentyitem{2021}{\textit{50 personal projects and collaborated in
      26.}}{GitHub}{My current
    {\href{https://github.com/BuddhiLW/}{\color{blue!50!violet}status
        on GitHub}} (08/2021)}
  % \twentyitem{2020}{\textit{Performance (Really) Matters}}{ACM}{Emery
  %   Berger on the use of Scalene to perform software optimization.}
  \twentyitem{2020}{\textit{International Congress on Funcional
      Programming}}{Penn University}{I learned about the state of the
    art on Programming Languages.}
  \twentyitem{2020}{\textit{Introduction to Git and
      GitHub}}{Google}{Part of the specialization, \textit{Google IT Automation with Python}}
  \twentyitem{2019}{\textit{Clojure for the Brave and True}}{Text Book}{Introductory self-study of Clojure.}
  \twentyitem{2018}{Arch Linux Install}{Linux Architecture}{I
    learned how to install Linux from scratch and configure a hole functional
    Desktop; used systemd, DWM as window manager and Emacs as my editor.}
  \twentyitem{2017}{Emacs, SLIME, Common Lisp}{Open Source}{When I
    started my interest on programming and Funcional Languages. I
    followed the book ``ANSI Common Lisp'', \textit{Paul Graham}.}
  \twentyitem{2017}{Ubuntu Linux}{First Install}{My objective
    was to use Emacs, which do not run smoothly on Windows.}
  % \twentyitem{<dates>}{<title>}{<location>}{<description>}
\end{twenty}

% ----------------------------------------------------------------------------------------
% OTHER INFORMATION
% ----------------------------------------------------------------------------------------
\vspace{0.2cm}
\section{Languages and Fluency}

\begin{description}
\item[Portuguese:] Native.
\item[English:] I read, speak and write at ease.
\item[Mandarin:] I read, speak and write at HSK1-2 level.
% \item[Latin:] I can read basic texts.
% \item[Espanhol:] Leio bem, falo pouco, escrevo pouco.
% \item[Francês:] Leio bem, falo pouco, escrevo pouco.

  % (\pinyin{Pi1nyin1})
% \item[Lojban:] Leio mediano, falo pouco, escrevo mediano.
\end{description}

\vspace{0.2cm}

% \section{Cultural Interests}

% \vspace{0.05mm}

% % \subsection{Filosofia}
% The most influential books for me were ``Science and Sanity',
% \textit{Alfred Korzibsky}; ``The Conquest of Happiness'', \textit{Bertrand Russell}; ``Discourse on the Origin and Basis of Inequality Among Men '',
% \textit{Jean Jacques Rousseau}.

% I have read almost throughout ``A Book of Set Theory'', \textit{Charles
%   Pinter}. And, ``The Qualitative Theory of Ordinary Differential
% Equations'', \textit{Fred Brauer, John A. Nohel}.
\end{document}

% ----------------------------------------------------------------------------------------
% SECOND PAGE EXAMPLE
% ----------------------------------------------------------------------------------------

% \newpage % Start a new page

% \makeprofile % Print the sidebar

% \section{Other information}

% \subsection{Review}

% Alice approaches Wonderland as an anthropologist, but maintains a strong sense of noblesse oblige that comes with her class status. She has confidence in her social position, education, and the Victorian virtue of good manners. Alice has a feeling of entitlement, particularly when comparing herself to Mabel, whom she declares has a ``poky little house," and no toys. Additionally, she flaunts her limited information base with anyone who will listen and becomes increasingly obsessed with the importance of good manners as she deals with the rude creatures of Wonderland. Alice maintains a superior attitude and behaves with solicitous indulgence toward those she believes are less privileged.

% \section{Other information}

% \subsection{Review}

% Alice approaches Wonderland as an anthropologist, but maintains a strong sense of noblesse oblige that comes with her class status. She has confidence in her social position, education, and the Victorian virtue of good manners. Alice has a feeling of entitlement, particularly when comparing herself to Mabel, whom she declares has a ``poky little house," and no toys. Additionally, she flaunts her limited information base with anyone who will listen and becomes increasingly obsessed with the importance of good manners as she deals with the rude creatures of Wonderland. Alice maintains a superior attitude and behaves with solicitous indulgence toward those she believes are less privileged.

% ----------------------------------------------------------------------------------------



%%% Local Variables:
%%% mode: latex
%%% TeX-master: t
%%% End:
