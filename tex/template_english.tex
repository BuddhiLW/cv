%%%%%%%%%%%%%%%%%%%%%%%%%%%%%%%%%%%%%%%%%
% Twenty Seconds Resume/CV
% LaTeX Template
% Version 1.1 (8/1/17)
%
% This template has been downloaded from:
% http://www.LaTeXTemplates.com
%
% Original author:
% Carmine Spagnuolo (cspagnuolo@unisa.it) with major modifications by
% Vel (vel@LaTeXTemplates.com)
%
% License:
% The MIT License (see included LICENSE file)
%
%%%%%%%%%%%%%%%%%%%%%%%%%%%%%%%%%%%%%%%%%

% ----------------------------------------------------------------------------------------
%	PACKAGES AND OTHER DOCUMENT CONFIGURATIONS
% ----------------------------------------------------------------------------------------

\documentclass[letterpaper]{../cls/twentysecondcvenglish} % a4paper for A4
\usepackage{hyperref}

% Definitions of Pinyin
\makeatletter
\def\py@yunpriv#1{%
  \if a#1 10\else
  \if o#1 9\else
  \if e#1 8\else
  \if i#1 7\else
  \if u#1 6\else
  \if v#1 5\else
  \if A#1 4\else
  \if O#1 3\else
  \if E#1 2\fi\fi\fi\fi\fi\fi\fi\fi\fi0
}

\def\py@init{%
  \edef\py@befirst{}%
  \edef\py@char{}\edef\py@tuneletter{}%
  \def\py@last{}%
  \def\py@tune{5}%
}

% Usage:
% \pinyin{Hao3hao3\ xue2xi2} (好好学习)
\def\pinyin#1{%
  \edef\py@postscan{#1}%
  \py@init
  % scan
  \loop
  \edef\py@char{\expandafter\@car\py@postscan\@nil}%
  \edef\py@postscan{\expandafter\@cdr\py@postscan\@nil}%
  \ifnum 0 < 0\py@char
    \edef\py@tune{\py@char}%
    \py@first \py@tuneat\py@tuneletter\py@tune \py@last\kern -4sp\kern 4sp{}\py@init
  \else
    \ifnum\py@yunpriv\py@char > \py@yunpriv\py@tuneletter
      \edef\py@tuneletter{\py@char}\edef\py@first{\py@befirst}\def\py@last{}%
    \else
      \edef\py@last{\py@last\if v\py@char\"u\else\py@char\fi}%
    \fi
    \edef\py@befirst{\py@befirst\if v\py@char\"u\else\py@char\fi}%
  \fi
  \ifx\py@postscan\@empty\else
  \repeat
}

\let\py@macron \=
\let\py@acute \'
\let\py@hacek \v
\let\py@grave \`

%% \py@tuneat{Letter}{tune}
\def\py@tuneat#1#2{%
  \if v#1%
    \py@tune@v #2%
  \else
  \if i#1%
    \py@tune@i #2%
  \else
    \ifcase#2%
      \or\py@macron #1\or\py@acute #1\or\py@hacek #1\or\py@grave #1\else #1%
    \fi
  \fi\fi
}

\def\py@tune@v#1{{%
    \dimen@ii 1ex%
    \fontdimen5\font 1.1ex%
    \rlap{\"u}%
    \fontdimen5\font .6ex%
    \ifcase#1%
      \or\py@macron u\or\py@acute u\or\py@hacek u\or\py@grave u\else u%
    \fi
    \fontdimen5\font\dimen@ii
  }}

\def\py@tune@i#1{%
  \ifcase#1
    \or\py@macron \i\or\py@acute \i\or\py@hacek \i\or\py@grave \i\else i%
  \fi
}
\makeatletter
% End of pinyin

% ----------------------------------------------------------------------------------------
% PERSONAL INFORMATION
% ----------------------------------------------------------------------------------------

% If you don't need one or more of the below, just remove the content leaving the command, e.g. \cvnumberphone{}

\profilepic{../img/W1-2} % Profile picture
\virtcv{../img/virtcv}
\cvname{Pedro Gomes \\Branquinho} % Your name
\cvjobtitle{Engineering Physicist} % Job title/career

\cvdate{October, 07, 1997} % Date of birth
\cvaddress{Franca, São Paulo - Brazil} % Short address/location, use \newline if more than 1 line is required
\cvnumberphone{+55 16 99340-1215} % Phone number
\linkedin{\href{https://www.linkedin.com/in/pedro-g-branquinho/}{LinkedIn Profile}} % Personal website
\cvsite{\href{www.buddhilw.com}{Personal Website}} % Personal website
% \cvsite2{\href{www.buddhilw.com}{\color{violet!60!black}www.buddhilw.com}} % Personal website
\cvmail{pedro.branquinho@usp.br} % Email address

% ----------------------------------------------------------------------------------------

\begin{document}

% ----------------------------------------------------------------------------------------
% ABOUT ME
% ----------------------------------------------------------------------------------------

\aboutme{I'm a Software Developer with strong background in STEM. I have know- ledge in both Front and Backend, as well as an advanced knowledge of Statistics and Scientific Computing.

  I have worked in Industry automating billing and reports for big and small companies. Also, I have profitable solo projects with Crypto Trading Bots.}

% To have no About Me section, just remove all the text and leave \aboutme{}

% ----------------------------------------------------------------------------------------
% SKILLS
% ----------------------------------------------------------------------------------------

% Skill bar section, each skill must have a value between 0 an 6 (float)
\skills{{Linux/5.0},{Front-end/5.0},{Back-end/4.5},{Scripting/4.0}, {Data Science/4.0}, {DevOps/3.5}, {Theoretical Computer Science/3.0}}
  % {Clojure(Script)/6}, {Emacs/5.7}, {LaTeX/5.7}, {Python-Julia-R/5.5}, {Go/5.5},{Git(Hub)/5.5}, {Scripting (Bash, et al.)/4}, {JavaScript/3.5}}
% ------------------------------------------------

% Skill text section, each skill must have a value between 0 an 6
% \skillstext{{lovely/4},{narcissistic/3}}

% ----------------------------------------------------------------------------------------

\makeprofile % Print the sidebar

% ----------------------------------------------------------------------------------------
% INTERESTS
% ----------------------------------------------------------------------------------------

\section{Professional Interests}

% \vspace{10mm}

\begin{itemize}
  \item Full-stack Development.
  \item Microservices.
  \item DevOps.
  \item Data Science.
  % \item Data Science - Recommender Systems, Hypothesis testing, PCA;
  % \item Webapp Applications using Machine Learning;
  % \item Linux System's automation and pipelines;
  % \item Backend Development;
  % \item Mathematical modeling of dynamic systems (ODEs/PDEs);
  % \item Operational Reaseach algorithms (OR-Tools);
\end{itemize}
\vspace{5mm}
% ----------------------------------------------------------------------------------------
% EDUCATION
% ----------------------------------------------------------------------------------------

\section{Formal Education}

\begin{twenty} % Environment for a list with descriptions
  \twentyitem{2016 - 2022}{Graduation (Completed)}{\normalsize{São Paulo University, USP}}{Engineering Physics}
  % \twentyitem{<dates>}{<title>}{<location>}{<description>}
\end{twenty}
\vspace{5mm}
% ----------------------------------------------------------------------------------------
% PUBLICATIONS
% ----------------------------------------------------------------------------------------

\section{Eletronic Publications}

\begin{twentyshort} % Environment for a short list with no
  % descriptions
  \twentyitemshort{2021}{\href{https://github.com/BuddhiLW/commons-csv-clj}{\color{blue!50!violet}
    Industry (Flow Finance) - Automate billing and Ledger history.}}
  \twentyitemshort{2021}{\href{https://github.com/BuddhiLW/CloshBashika}{\color{blue!50!violet}
    Industry (Lupo S.A.) - Automate technical reports with Clojure/LaTeX.}}
\end{twentyshort}
\vspace{0.2cm}
% ----------------------------------------------------------------------------------------
% Industry Experience
% ----------------------------------------------------------------------------------------

\vspace{5mm}
\section{Jobs -- Industry and Academia}

\begin{twenty} % Environment for a short list with no descriptions
  % \twentyitem{<dates>}{<title>}{<location>}{<description>}
  \twentyitem{2021}{Lupo S.A. -- Jan/2021-May/2021}{Developer}{Wrote software to automate Technical Reports, while working at WJB Engenharia (wjbsegurançadotrabalho.com.br/), as a Contractor firm to Lupo S.A.. We performed the Safety Analysis and Inventory of all the machinery from the company.}
  \twentyitem{2021}{FlowFinance S.C.- Jun/2021-July/2021}{Developer}{I developed an application, single-handily, to perform the Ledger and Clarence of billing data coming from BIORC, in CSV format. The technology used was Clojure.}
  \twentyitem{2021/2022}{University of São Paulo - Sep/2021-Mar/2022}{Researcher}{Modeling Traffic Flow, with Julia and Python - numerical solution to Partial Differential Equations (PDEs).Modeling Traffic Flow, with Julia and Python - numerical solution to Partial Differential Equations (PDEs).
}
\twentyitem{2021/2022}{Café do Bem (NPO) - Aug/2021-Current}{Volunteer (Free time)}{I created the website \href{https://cafe-do-bem.company.site/}{https://cafe-do-bem.company.site/}, which is a platform to sell coffee. The mission of this Non-profit Organization is to revert all monetary gain, back to the coffee community (housing, food and basic education, as well as specialization courses).
}

\twentyitem{2022/2023}{FACTI - Dez/2022-Aug/2023}{Developer}{
Working on an application to facilitate the accountability of projects, which have been funded by the government. The technologies used are Clojure, Vanilla JavaScript, Angular, JQuery, Express.js, Bootstrap and Material UI.
}
\end{twenty}

\vspace{5mm}
% ----------------------------------------------------------------------------------------
% LANGUAGE
% ----------------------------------------------------------------------------------------
\section{Languages and Fluency}

\begin{description}
\item[Portuguese:] Native.
\item[Inglês:] \href{https://www.efset.org/cert/hqg62J}{C2 Level} (click to open certificate).
\item[Mandarim:] HSK2.
\end{description}

\vspace{0.2cm}

% \section{Cultural Interests}

% \vspace{0.05mm}

% % \subsection{Filosofia}
% The most influential books for me were ``Science and Sanity',
% \textit{Alfred Korzibsky}; ``The Conquest of Happiness'', \textit{Bertrand Russell}; ``Discourse on the Origin and Basis of Inequality Among Men '',
% \textit{Jean Jacques Rousseau}.

% I have read almost throughout ``A Book of Set Theory'', \textit{Charles
%   Pinter}. And, ``The Qualitative Theory of Ordinary Differential
%   Equations'', \textit{Fred Brauer, John A. Nohel}.


% ----------------------------------------------------------------------------------------
% SECOND PAGE EXAMPLE
% ----------------------------------------------------------------------------------------

\newpage % Start a new page

\makeprofileNoExtra % Print the sidebar

\section{\LARGE{Certificates}}
\subsection{\textbf{Google Go Specialization by UCI - Irvine}}
\begin{description}
\item[\href{https://buddhilw.github.io/bug-free-fiesta/}{Getting Started with Go:}] Basic syntax and Data Structure.
\item[\href{https://buddhilw.github.io/bug-free-fiesta/}{Functions, Methods and Interfaces:}] Object Orientation in Go.
\item[\href{https://buddhilw.github.io/bug-free-fiesta/}{Concurrency in Go:}] Advanced features in Go.
\end{description}
\vspace{10mm}
\subsection{\textbf{Backend by Meta (former Facebook)}}
\begin{description}
\item[\href{https://buddhilw.github.io/bug-free-fiesta/}{Introduction to Back-End Development:}] Agile, TDD etc.; the infrastructure of the web, HTTP, CTP and other Protocols..
\item[\href{https://buddhilw.github.io/bug-free-fiesta/}{Version Control:}] Practices with Git and Github.
\item[\href{https://buddhilw.github.io/bug-free-fiesta/}{Programming in Python:}] Syntax, and Object Orientation.
\end{description}

\vspace{10mm}
\subsection{\textbf{Responsive Web Design by freeCodeCamp}}

\begin{description}
\item[\href{https://buddhilw.github.io/bug-free-fiesta/}{Responsive Web Design:}] CSS syntax; Selectors; Bootstrap and other Frameworks.
\end{description}

\vspace{10mm}
\subsection{\textbf{Introduction to Docker: Build Your Own Portfolio Site by Coursera}}

\begin{description}
\item[\href{https://buddhilw.github.io/bug-free-fiesta/}{Introduction to Docker: Build Your Own Portfolio Site:}] Launch a web-application using Docker; learn to download, upload and use containers.
\end{description}

\vspace{10mm}
\subsection{\textbf{English certification by EF Education First}}

\begin{description}
\item[\href{https://buddhilw.github.io/bug-free-fiesta/}{EF SET Score - C2}] English test, based on European Standards - 78 points, giving the C2 score, at the maximum-range.
\end{description}


\vspace{5mm}
\section{Aggregate-website, with Certifications}

\vspace{1mm}

\begin{center}
  Link: \href{https://buddhilw.github.io/bug-free-fiesta/}{https://buddhilw.github.io/bug-free-fiesta/}
\vspace{1mm}

\includegraphics[width=\imagewidth]{../img/virtcerts.png}
\end{center}
\newpage
\makeprofileNoExtra

\section{Eletronic Publications}

\begin{twentyshort} % Environment for a short list with no
  % descriptions
  \twentyitemshort{2020}{\href{https://github.com/BuddhiLW/EEL_EstatMultiVarRN}{\color{blue!50!violet}Machine
    Learning, with R - while enrolled in Multivariate Statics.}}
  \twentyitemshort{2020}{\href{https://github.com/BuddhiLW/MC-LaTeX/tree/master/LabEELw/LabEELw/MaterialMC/Livro_PT1}{\color{blue!50!violet}Introduction
      to \LaTeX, P1.}}
  \twentyitemshort{2020}{\href{https://github.com/BuddhiLW/MC-LaTeX/tree/master/LabEELw/LabEELw/MaterialMC/Livro_PT2}{\color{blue!50!violet}
      Essential Commands in \LaTeX, P2.
    }}
  \twentyitemshort{2020}{\href{https://github.com/BuddhiLW/MC-LaTeX/tree/master/LabEELw/LabEELw/MaterialMC/Livro_PT3}{\color{blue!50!violet}Bibliographical
      References, Citations and Beamer Presentations, P3.}}
  \twentyitemshort{2020}{\href{https://www.youtube.com/watch?v=spYCKElN-v0&list=PLweBgessvalf9eGUknfHQZ6_q2Nz7t8df}{\color{blue!50!violet}
    A Minicourse on \LaTeX{}} - Material available on YouTube (Portuguese).}
\end{twentyshort}

\vspace{10mm}
\section{Extracurricular Experiences - While at University}

\begin{twenty} % Environment for a list with descriptions
  % \twentyitem{<dates>}{<title>}{<location>}{<description>}
  \twentyitem{2022}{\textit{Programming in Python}}{Meta}{Course on Algorithms and Datastructures using Python.}
  \twentyitem{2022}{\textit{Version Control}}{Meta}{Course on Git, Github and Version Control practices.}
  \twentyitem{2022}{\textit{Introduction to Back-End Development}}{Meta}{Course on how the Web Works, HTTP, REST, APIs, TDD and best practices.}
  \twentyitem{2022}{\textit{Google Go Specialization}}{University of California (UCI)}{Three courses exploring Golang syntax, Object Orientation and Concurrency}
  \twentyitem{2021}{\textit{50 personal projects and collaborated in
      26.}}{GitHub}{My current
    {\href{https://github.com/BuddhiLW/}{\color{blue!50!violet}status
        on GitHub}} (08/2021)}
  \twentyitem{2020}{\textit{Performance (Really) Matters}}{ACM}{Emery
    Berger on the use of Scalene to perform software optimization.}
  \twentyitem{2020}{\textit{International Congress on Funcional
      Programming}}{Penn University}{I learned about the state of the
    art on Programming Languages.}
  \twentyitem{2020}{\textit{Introduction to Git and
      GitHub}}{Google}{Part of the specialization, \textit{Google IT Automation with Python}}
  \twentyitem{2019}{\textit{Clojure for the Brave and True}}{Text Book}{Introductory self-study of Clojure.}
  \twentyitem{2018}{Arch Linux Install}{Linux Architecture}{I
    learned how to install Linux from scratch and configure a hole functional
    Desktop; used systemd, DWM as window manager and Emacs as my editor.}
  \twentyitem{2017}{Emacs, SLIME, Common Lisp}{Open Source}{When I
    started my interest on programming and Funcional Languages. I
    followed the book ``ANSI Common Lisp'', \textit{Paul Graham}.}
  \twentyitem{2017}{Ubuntu Linux}{First Install}{My objective
    was to use Emacs, which do not run smoothly on Windows.}
\end{twenty}

\end{document}

%%% Local Variables:
%%% mode: latex
%%% TeX-master: t
%%% End:
