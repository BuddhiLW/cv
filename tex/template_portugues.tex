%%%%%%%%%%%%%%%%%%%%%%%%%%%%%%%%%%%%%%%%%
% Twenty Seconds Resume/CV
% LaTeX Template
% Version 1.1 (8/1/17)
%
% This template has been downloaded from:
% http://www.LaTeXTemplates.com
%
% Original author:
% Carmine Spagnuolo (cspagnuolo@unisa.it) with major modifications by
% Vel (vel@LaTeXTemplates.com)
%
% License:
% The MIT License (see included LICENSE file)
%
%%%%%%%%%%%%%%%%%%%%%%%%%%%%%%%%%%%%%%%%%

% ----------------------------------------------------------------------------------------
%	PACKAGES AND OTHER DOCUMENT CONFIGURATIONS
% ----------------------------------------------------------------------------------------

\documentclass[letterpaper]{../cls/twentysecondcv} % a4paper for A4
\usepackage{hyperref}

% Definitions of Pinyin
\makeatletter
\def\py@yunpriv#1{%
  \if a#1 10\else
  \if o#1 9\else
  \if e#1 8\else
  \if i#1 7\else
  \if u#1 6\else
  \if v#1 5\else
  \if A#1 4\else
  \if O#1 3\else
  \if E#1 2\fi\fi\fi\fi\fi\fi\fi\fi\fi0
}

\def\py@init{%
  \edef\py@befirst{}%
  \edef\py@char{}\edef\py@tuneletter{}%
  \def\py@last{}%
  \def\py@tune{5}%
}

% Usage:
% \pinyin{Hao3hao3\ xue2xi2} (好好学习)
\def\pinyin#1{%
  \edef\py@postscan{#1}%
  \py@init
  % scan
  \loop
  \edef\py@char{\expandafter\@car\py@postscan\@nil}%
  \edef\py@postscan{\expandafter\@cdr\py@postscan\@nil}%
  \ifnum 0 < 0\py@char
    \edef\py@tune{\py@char}%
    \py@first \py@tuneat\py@tuneletter\py@tune \py@last\kern -4sp\kern 4sp{}\py@init
  \else
    \ifnum\py@yunpriv\py@char > \py@yunpriv\py@tuneletter
      \edef\py@tuneletter{\py@char}\edef\py@first{\py@befirst}\def\py@last{}%
    \else
      \edef\py@last{\py@last\if v\py@char\"u\else\py@char\fi}%
    \fi
    \edef\py@befirst{\py@befirst\if v\py@char\"u\else\py@char\fi}%
  \fi
  \ifx\py@postscan\@empty\else
  \repeat
}

\let\py@macron \=
\let\py@acute \'
\let\py@hacek \v
\let\py@grave \`

%% \py@tuneat{Letter}{tune}
\def\py@tuneat#1#2{%
  \if v#1%
    \py@tune@v #2%
  \else
  \if i#1%
    \py@tune@i #2%
  \else
    \ifcase#2%
      \or\py@macron #1\or\py@acute #1\or\py@hacek #1\or\py@grave #1\else #1%
    \fi
  \fi\fi
}

\def\py@tune@v#1{{%
    \dimen@ii 1ex%
    \fontdimen5\font 1.1ex%
    \rlap{\"u}%
    \fontdimen5\font .6ex%
    \ifcase#1%
      \or\py@macron u\or\py@acute u\or\py@hacek u\or\py@grave u\else u%
    \fi
    \fontdimen5\font\dimen@ii
  }}

\def\py@tune@i#1{%
  \ifcase#1
    \or\py@macron \i\or\py@acute \i\or\py@hacek \i\or\py@grave \i\else i%
  \fi
}
\makeatletter
% End of pinyin

% ----------------------------------------------------------------------------------------
% PERSONAL INFORMATION
% ----------------------------------------------------------------------------------------

% If you don't need one or more of the below, just remove the content leaving the command, e.g. \cvnumberphone{}

\profilepic{../img/W1-2} % Profile picture
\virtcv{../img/virtcv}
\cvname{Pedro Gomes \\Branquinho} % Your name
\cvjobtitle{Engenheiro Físico} % Job title/career
\cvdate{07 Outubro 1997} % Date of birth
\cvaddress{Franca, SP} % Short address/location, use \newline if more than 1 line is required
\cvnumberphone{(16) 99340-1215} % Phone number
\linkedin{\href{https://www.linkedin.com/in/pedro-g-branquinho/}{Perfil LinkedIn}} % Personal website
\cvsite{\href{www.buddhilw.com}{www.buddhilw.com}} % Personal website
\cvmail{pedro.branquinho@usp.br} % Email address

% ----------------------------------------------------------------------------------------

\begin{document}

% ----------------------------------------------------------------------------------------
% ABOUT ME
% ----------------------------------------------------------------------------------------

\aboutme{Sou um Desenvolvedor de Software, com forte formação em exatas. Tenho conhecimento em ambos Front e Back- end, bem como conhecimento avançado de Estatística e Computação Científica.

  Tenho experiência free-lance na Indústria, em empresas de pequeno e grande porte.
}

% To have no About Me section, just remove all the text and leave \aboutme{}

% ----------------------------------------------------------------------------------------
% SKILLS
% ----------------------------------------------------------------------------------------
\skills{{Emacs/5.7}, {LaTeX/5.7},  {Python/5.5}, {Go/5},{Julia/4.5},{Git/4.5},{Clojure(Script)/4.5}, {Scripting (Bash, et al.)/4}, {JavaScript/3.5}}

% ----------------------------------------------------------------------------------------

\makeprofile % Print the sidebar

% ----------------------------------------------------------------------------------------
% INTERESTS
% ----------------------------------------------------------------------------------------

\section{Interesses Profissionais}

% \vspace{10mm}

\begin{itemize}
  \item Data Science - Teste de Hipótese, Sistemas de Recomendação, PCA;
  \item Aplicações Web, usando Machine Learning;
  \item Automação de pipelines Linux/Unix;
  \item Desenvolvimento Backend;
\end{itemize}
\vspace{0.2cm}
% ----------------------------------------------------------------------------------------
% EDUCATION
% ----------------------------------------------------------------------------------------

\section{Educação}

\begin{twenty} % Environment for a list with descriptions
  \twentyitem{2016 - 2022}{Graduação (Completa)}{\normalsize{Universidade de São Paulo, USP}}{Engenharia Física}
  % \twentyitem{<dates>}{<title>}{<location>}{<description>}
\end{twenty}
\vspace{0.2cm}
% ----------------------------------------------------------------------------------------
% PUBLICATIONS
% ----------------------------------------------------------------------------------------

\section{Publicações Eletônicas}

\begin{twentyshort} % Environment for a short list with no
  % descriptions
  \twentyitemshort{2021}{\href{https://github.com/BuddhiLW/commons-csv-clj}{\color{blue!50!violet}
    Indústria (Flow Finance) - Automação de Boletos}}
  \twentyitemshort{2021}{\href{https://github.com/BuddhiLW/CloshBashika}{\color{blue!50!violet}
    Industria (Lupo S.A.) - Automação de Relatórios Técnicos com Clojure/LaTeX.}}
  \twentyitemshort{2020}{\href{https://github.com/BuddhiLW/EEL_EstatMultiVarRN}{\color{blue!50!violet}Machine
    Learning, com R - Enquanto cursava Estatística Multivareada.}}
  \twentyitemshort{2020}{\href{https://www.youtube.com/watch?v=spYCKElN-v0&list=PLweBgessvalf9eGUknfHQZ6_q2Nz7t8df}{\color{blue!50!violet}
    Um Minicurso em \LaTeX{}} - Disponível no YouTube.}
\end{twentyshort}
\vspace{0.2cm}
% ----------------------------------------------------------------------------------------
% AWARDS
% ----------------------------------------------------------------------------------------

% \section{Awards}

% \begin{twentyshort} % Environment for a short list with no descriptions
%   \twentyitemshort{1987}{All-Time Best Fantasy Novel.}
%   \twentyitemshort{1998}{All-Time Best Fantasy Novel before 1990.}
%   % \twentyitemshort{<dates>}{<title/description>}
% \end{twentyshort}

% ----------------------------------------------------------------------------------------
% EXPERIENCE
% ----------------------------------------------------------------------------------------

\section{Experiencias Extracurriculares}

\begin{twenty} % Environment for a list with descriptions
  \twentyitem{2022}{\textit{Programming in Python}}{Meta}{Curso sobre Algorítmos e Estrtura de dados.}
  \twentyitem{2022}{\textit{Version Control}}{Meta}{Curso em Git, Github e melhor práticas, em Versionamento.}
  \twentyitem{2022}{\textit{Introduction to Back-End Development}}{Meta}{Curso de introdução às melhores práticas, TDD, Git e Metodologias Ágeis.}
  \twentyitem{2022}{\textit{Google Go Specialization}}{University of California (UCI)}{Profissionalização em Go/Golang: Programação Orientada a Objetos (POO), Paralelismo, Concorrência e Design de Softwares.}
  \twentyitem{2021}{\textit{Marco: 50 projetos pessoas e 26 colaborações (PR)}}{GitHub}{Meu atual
    {\href{https://github.com/BuddhiLW/}{\color{blue!50!violet}status no GitHub}} (08/2021)}
  % \twentyitem{2020}{\textit{Performance (Really) Matters}}{ACM}{Emery
  %   Berger on the use of Scalene to perform software optimization.}
  \twentyitem{2020}{\textit{International Congress on Funcional
      Programming (ICFP))}}{Penn University}{Maior congresso em linguagens funcionais - exposição ao estado da arte da computação.}
  \twentyitem{2020}{\textit{Introduction to Git and
      GitHub}}{Google}{Parte da especialização, \textit{Google IT Automation with Python}}
  \twentyitem{2019}{\textit{Clojure for the Brave and True}}{Livro Texto}{Início do auto-aprendizado de Clojure.}

  \twentyitem{2018}{Instalação do Arch Linux}{Arquitetura Linux}{Aprendizado de como configurar um Desktop, do zero. Experiência com APIs, integrações e Third-Party-Software.}

  \twentyitem{2017}{Emacs, SLIME, Common Lisp}{Open Source}{Quando comecei a estudar progamação, algoritmos funcionais e design de softwares.}

  \twentyitem{2017}{Ubuntu Linux}{Primeira Instalação}{Primeira experiência com Linux, SystemD e Shell Scripting.}
  % \twentyitem{<dates>}{<title>}{<location>}{<description>}
\end{twenty}

% ----------------------------------------------------------------------------------------
% OTHER INFORMATION
% ----------------------------------------------------------------------------------------
\vspace{0.2cm}
\section{Línguas e Fluência}

\begin{description}
\item[Inglês:] \href{https://www.efset.org/cert/hqg62J}{Fluência a nível E2} (clique para abrir o certificado).
\item[Mandarim:] HSK2.
\end{description}

\vspace{0.2cm}


% ----------------------------------------------------------------------------------------
% SECOND PAGE EXAMPLE
% ----------------------------------------------------------------------------------------

\newpage % Start a new page

\makeprofileNoExtra % Print the sidebar

\section{\LARGE{Certificados}}
\subsection{\textbf{Google Go Specialization}}




% -----------------------------------------------------------------------------



\end{document}
%%% Local Variables:
%%% mode: latex
%%% TeX-master: t
%%% End:

